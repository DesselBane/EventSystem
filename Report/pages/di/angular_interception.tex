% !TEX root=../../report.tex

\Section[ngInterception]{Interception mit Angular}
Seit Angular in der Version 4.3 erschienen ist, gibt es eine neue Bibliothek für den
Umgang mit gls{http}-Requests. Um Stichtagsmigrationen zu vermeiden, wird sowohl das bisherige HttpClientModule, als auch der neue HttpClient unterstützt.
Die Angularversion des EventManagers musste auf die Version 4.3 angehoben werden, da nur der neue HttpClient die Nutzung von Interceptoren unterstützt.


\Subsubsection{Migration von HttpClientModule  zur HttpClient Bibliothek}
Bei der Umstellung auf die neue Bibliothek müssen alle Stellen im Client angepasst werden, die \gls{http}-Anfragen versenden.
Hauptsächlich betrifft dies die Services, da diese die einzigen Klassen sind, welche \gls{http}-Anfragen versenden.

Zunächst muss der Import von HttpClientModule zu HttpClient angepasst werden, welches nun aus dem Paket \enquote{@angular/common/http} statt \enquote{@angular/http} stammt.
Außerdem muss nun die neue HttpClient Klasse, anstelle der bisherigen http Klasse, injiziert werden.
\cite{httpClient} \cite{httpInter}.

\Subsubsection{Verwendung von Interceptoren im Projekt} \label{authInterceptor}
Die Interceptoren werden im EventManager zum Setzen des \enquote{Content-Type} Attributes in den Headern der ausgehenden \gls{http}-Requests verwendet.
Dies geschieht, wie in \myautoref{lst:authenticationinterceptor.ts} zu sehen ist, in einem eigenen \enquote{AuthenticationInterceptor}, welcher außerdem den Authorization-Header setzt. Da diese Funktionalität bei jedem Request benötigt wird, kann sie in einem Interceptor besonders gut gekapselt werden.

\file{authenticationinterceptor.ts}{AuthenticationInterceptor des EventManagers}{JavaScript}
