% !TEX root=../../report.tex

\section{Kommunikation im Team}

Um die möglichen Vorteile einer solchen Teamaufteilung auszuschöpfen, ist ein reger Austausch zwischen den Teammitgliedern zwingend erforderlich.

Ein großer Teil davon wird über die Arbeitsaufgaben abgehandelt. Diese werden in \myref{wi} näher beschrieben.

Wöchentliche Statusrunden, oder auch Sitzungen zur Problemlösung werden über die Kommunikationsplattform Discord erledigt. Es unterstützt sowohl die Kommunikation über Voice-over-IP, als auch den textuellen Austausch. Die Textnachrichten können nach Themen gruppiert und wichtige Informationen präsent angeboten werden.

Die Absprache per WhatsApp hat sich nicht bewährt, da dort viele Informationen überlesen werden, oder einfach in Vergessenheit geraten.

\subsubsection{Dokumentation und Anleitungen}

Für handschriftliche Notizen und Skizzen wird ein OneNote-Notitzbuch angelegt, auf das alle Teammitglieder Zugriff haben. Für die Entwicklung relevante Anleitungen werden im GitLab der Fakultät Informatik gespeichert. Dafür eignet sich der Bereich \enquote{Wiki}. Die API-Dokumentation des Backends wird mit dem Swagger Framework umgesetzt, welches in \myref{sec:swagger} beschrieben wird.
