% !TEX root=../../report.tex

\section{Verwaltung der Arbeitsaufgaben}
\label{wi}

Für die Dokumentation und Verteilung der zu erledigenden Aufgaben wird der Issue-Tracker des GitLabs der Fakultät Informatik verwendet. Dort werden alle Arbeiten gesammelt und eingeplant. Das Tool erleichtert die Koordination von Arbeiten und ermöglicht es, den Projektstatus abzulesen.

Zur besseren Übersicht werden die Arbeitsaufgaben in Meilensteine eingeteilt, an deren Ende alle geplanten Aufgaben erledigt und getestet sein sollen. Die Meilensteine werden auf die Zeitspanne zwischen den Berichten an Herr Professor Teßmann festgelegt.

Aufgrund der Präferenzen im Team wird jede Arbeitsaufgabe mit entsprechenden Tags versehen, je nachdem, ob es sich um einen Fehler, ein Feature des Frontends, oder des Backends handelt. Außerdem kann vermerkt werden, ob noch Zuarbeiten anderer Teammitglieder erforderlich sind.
