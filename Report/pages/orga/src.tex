% !TEX root=../../report.tex

\section{Sourcecodeverwaltung}

Der Programmcode wird im Repository des GitLabs der Fakultät Informatik gespeichert und versioniert.

Das Vorgehen, bei dem alle Beteiligten auf den \enquote{master branch} pushen wird aufgrund von häufigen Merge-Konflikten geändert. Bei solchen Konflikten werden beim Pushen der eigenen Arbeiten die eines anderen Teammitglieds überschrieben. Dadurch lässt sich der aktuelle Stand oft nicht kompilieren und der Fehler muss in mühsamer Teamarbeit gefunden und so behoben werden, dass alle eingebrachten Features noch funktionieren.

Anpassungen wurden vorgenommen, damit für jede Arbeitsaufgabe im Issue-Tracker ein eigener Branch angelegt wird. Alle Arbeitsschritte, die zur Erledigung einer Aufgabe führen, werden ausschließlich auf diesen Branches erledigt, ebenso wie der abschließende Nachtest dieser Aufgabe. Ist der Test erfolgreich, so wird der Branch auf den master gemerged. Das neue Vorgehen führt zu weniger Konflikten und somit zu einer höheren Qualität des Systems.
