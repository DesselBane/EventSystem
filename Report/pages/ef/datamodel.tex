% !TEX root=../../report.tex

\Section{Datenmodell}

Das Datenmodell (\myautoref{fig:datamodel.jpg}) beinhaltet alle Domain-Klassen des EventManagers. Hier ist zu sehen, dass zwischen den Objekttypen \lstcode{User} und \lstcode{Person} unterschieden wird. Dies hat den Grund, dass ein Administrator des Systems nicht zwangsweise auch ein Nutzer sein muss. Des Weiteren besteht die Möglichkeit, einen \enquote{Dummynutzer} zu erstellen, sollte man Gäste zu einem Event hinzufügen wollen, welche keinen Account im System besitzen. Dieses Feature ist im Moment noch nicht implementiert, ist jedoch als Erweiterung angedacht. Jeder \lstcode{User} kann viele Claims haben. Diese Claims enthalten Informationen über den Nutzer, wie \zB die EMail Adresse, oder aber auch Berechtigungen wie die des \enquote{ServiceTypeAdministrators}.

\autoImg{datamodel.jpg}{Datenmodell des EventManagers}

Die wichtigsten Klassen sind die \lstcode{Event}, \lstcode{ServiceType}, \lstcode{ServiceSlot}, \lstcode{Service\-Agreement} sowie \lstcode{EventService} Klassen. Dies sind die Datenstrukturen, welche alle Daten über Dienstleister sowie Events speichern. Jedes Event kann mehrere \lstcode{ServiceSlot} Objekte referenzieren. Ein \lstcode{ServiceSlot} ist der Platzhalter für eine bestimmte Dienstleistung. So möchte \zB ein Eventplaner einen DJ für sein Event buchen. Im ServiceSlot wird festgelegt, was für eine Art von Service gewünscht ist, in welchem Zeitraum dieser Service gebraucht wird, und was dieser kosten soll. Ist der \lstcode{ServiceSlot} erstellt, kann der Eventplaner sich einen Dienstleister (\lstcode{EventService}) suchen. Hat er einen passenden Dienstleister gefunden, wird ein sogenanntes \lstcode{ServiceAgreement} erstellt. In diesem \lstcode{ServiceAgreement} werden ähnliche Werte gespeichert wie im \lstcode{ServiceSlot}. Damit könnte man später eine Soll-Ist-Analyse machen, diese ist momentan aber noch nicht implementiert. Das \lstcode{ServiceAgreement} hat zusätzlich noch eine \lstcode{State} Property. Diese Property ist vom Typ \lstcode{ServiceAgreementStates}, welcher in der Datenbank als Integer abgebildet wird. Dieses Enum kann die Werte \lstcode{Requested}, \lstcode{Proposal}, \lstcode{Accepted} sowie \lstcode{Declined} annehmen.

Die \lstcode{Specification} und \lstcode{Attribute} Klassen sind als nächstes Feature geplant. Diese sind bereits im Datenmodell vorhanden, allerdings fehlen noch die Funktionalitäten im Front- und Backend. Für verschiedene Dienstleistungen sind verschiedene Daten relevant. So möchte ein Nutzer \zB bei einer Hallenvermietung wissen, wie groß die zu vermietende Halle ist, wobei die gleiche Frage bei einem DJ sinnlos wäre. Im EventManager System gibt es bereits den \lstcode{ServiceType}, um eine Dienstleistung zu beschreiben. Jeder \lstcode{ServiceType} kann mehrere \lstcode{Specification} Objekte referenzieren. Ein \lstcode{Specification} Objekt speichert, welche Daten ein Dienstleister jeweils angeben muss. Um diese Spezifikationen, welche für jeden Dienstleister unterschiedlich sind, mit konkreten Werten zu füllen, gibt es die \lstcode{Attribute} Objekte. So muss der Dienstleister, der eine Halle vermieten will, ein Attribut erstellen welches \zB den Wert $100 m^2$ enthält und die Spezifikation \enquote{Größe} referenziert. Das gleiche Prinzip ist auch für die \lstcode{ServiceAgreement} Objekte vorgesehen.
