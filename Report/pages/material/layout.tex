% !TEX root=../../report.tex

\Section{Layout}

Die Layoutmöglichkeiten in Angular Material haben in der neuen Version für Angular (oft auch Angular2 genannt) leider abgenommen.
Die Vorgängerversion für AngularJS bot ein umfangreiches Layout \gls{api} an, mit dem es möglich war, die Anordnung der Bedienelemente je nach
Displaygröße festzulegen. Grundlegend hierbei ist die Einteilung der Bildschirmgrößen in festgelegte Sektionen von xsmall (unter 480 Pixel Breite) bis xlarge (ab 1920 Pixel Breite).

Einige wichtige Aspekte beim Umsetzen eines Oberflächenlayouts in \gls{html} und \gls{css} werden in \myautoref{fig:fxlayoutgroessen.png} beschrieben.

\autoImg{fxlayoutgroessen.png}{Übersicht über die Einteilung der Bildschirmgrößen}

\Subsubsection{Verwendung von FxLayout}
\image[scale=.5]{mediaquerie.png}{Beispiele verschiedener MediaQueries \cite{mediaQueries}}

FxLayout ist eine Layout \gls{api} für Angular, die sich stark an der Funktions- und Verwendungsweise der ursprünglichen Gestaltungsmöglichkeiten von AngularJS orientiert.
Für die Oberflächengestaltung werden \gls{html}-Tags von FxLayout verwendet. Diese werden beim Build-Prozess über die AngularCLI in Flexbox \gls{css}-Attribute und \gls{css}-MediaQueries übersetzt.

\Subsubsection{Zeilen und Spaltenorientierung}

Bei Layouts wird generell zwischen zeilen- und spaltenorientierten Layouts unterschieden. Oft wird eine beliebig verschachtelte Kombination aus beiden Orientierungen verwendet. Wie in \myautoref{fig:profilansicht.png} zu sehen ist, verwendet die Profilkachel eine Spaltenorientierung, wohingegen ihre Kindelemente zeilenartig angeordnet sind.

\autoImg{profilansicht.png}{Das Layout der Profilansicht}

\Subsubsection{Inhaltsausrichtung}
Die Ausrichtung des Inhaltes kann mit FxLayout-Attributen umgesetzt werden, deren Namensgebung sich stark an denen der \gls{css}-Eigenschaften des \enquote{justify-content} orientiert.
Zu beachten ist, dass sich der verfügbare Platz nach der Größe des Elternelements richtet. Es ist also sorgfältig darauf zu achten, wie Elemente verschachtelt sind und welche
Attribute jeweils gesetzt sind, da diese an Kindelemente vererbt werden.
