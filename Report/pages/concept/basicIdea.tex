% !TEX root=../../report.tex

\Section{Grundidee}

Die Grundidee des EventManagers ist es, eine Plattform zur Verfügung zu stellen. Mit dieser sollen sich Veranstaltungen, wie zum Beispiel Hochzeiten, Geburtstage oder andere größere Events planen lassen. Zur erleichterten Planung stellt die Anwendung sogenannte \enquote{Serviceslots} bereit. Diese Slots werden mit Kategorien ausgestattet, um so einen Überblick zu haben, welche Dienstleistungen eingeplant sind und welche noch fehlen könnten. Weiterhin wird über die angezeigten Informationen der zeitliche Ablauf des Events abgebildet. Auch können die Kosten leicht abgelesen werden. Diese Slots können von dem Veranstalter mit den passenden Anbietern gefüllt werden. Ausgewählt werden sie über eine Suche, die automatisch nach den Kategorien filtert. Durch das Hinzufügen eines Dienstleisters erhält dieser automatisch eine Anfrage. Mithilfe der Plattform kann dann über die Rahmenbedingungen, wie Uhrzeiten oder die Bezahlung, verhandelt werden. Sind beide Partner einverstanden, kommt der Vertrag zustande und der Dienstleister ist für die Veranstaltung gebucht. Soll auf einer Party beispielsweise Musik gespielt werden, wird ein Serviceslot für einen Musiker angelegt. Wird dort jetzt nach Dienstleistern gesucht, werden nur musikalische Leistungen angeboten.

Zu einer geplanten Veranstaltung sollen auch Gäste eingeladen werden können. Diese können über die Webseite die Veranstaltung sowie ihre Rahmenbedingungen einsehen und auch zu- oder absagen. Somit soll eine Planung über die Personenzahl möglich werden. Der Vorteil hierbei ist, dass die notwendigen Schritte einer Eventplanung auf einer Plattform erledigt werden können.

Um auch eine Auswahl an Dienstleistern zur Verfügung stellen zu können, sollen sich verschiedene Anbieter registrieren können. Über einen extra Menüpunkt soll dann die Dienstleistung, inklusive individueller Beschreibung und Preisvorstellung, angeboten werden können.
