% !TEX root=../../report.tex

\Section{Weiterführende Anforderungen}

Die in \myautoref{ucSection} beschriebenden Anwendungsfälle bilden Anforderungen an das System ab, die auf jeden Fall umgesetzt werden müssen. In einer Diskussionsrunde werden allerdings nicht nur solche Muss-Anforderungen diskutiert. Es gibt auch viele weiterführende Ideen, die die Handhabung an einigen Stellen erweitern, beziehungsweise erleichtern. Auch diese werden gesammelt.

\subsubsection{Notification-System}

Optional kann ein sogenanntes \enquote{Notification-System} entwickelt werden. Der Hintergrund dieser Überlegung ist, dass ein Benutzer die Seite bei jeder Änderung, die nicht durch ihn erwirkt wird, neu laden muss. Das kann zum Beispiel sein, wenn er zu einem neuen Event eingeladen wird. Jedes neue Event wird beim angemeldeten Benutzer erst in der Liste erscheinen, wenn die Seite sich die neuen Daten geholt hat. Dies geschieht, indem die Seite neu geladen wird.

Es wird jedoch an keiner Stelle und zu keinem Zeitpunkt gemeldet, wann oder ob dies vonnöten ist. Diese Aufgabe kann das Notification-System übernehmen, das regelmäßig auf aktualisierte Informationen prüft und dem Benutzer entsprechend Rückmeldung gibt.

Diese Erweiterung kann etwaige Verwirrungen beim Anwender auflösen, warum er zum Beispiel Informationen nicht zum erwarteten Zeitpunkt erhält. Es trägt zur Verbesserung der Bedienung und der Verständlichkeit bei.

\subsubsection{Chat-Funktion}

Verhandeln ein Veranstalter und ein Anbieter über eine Dienstleistung für eine Veranstaltung, so können sie nur die vorgegebenen Felder für die Beschreibung und den Preis verwenden. Ist dies nicht ausreichend, so müssen beide auf ein externes Kommunikationssystem zurückgreifen.

In diesem Fall kann es die Verhandlung erleichtern, wenn ein \enquote{Chat-System} zur Verfügung gestellt wird. So können sich beide Partner über dieselbe Plattform austauschen, auf der die Vereinbarung dokumentiert wird.

\subsubsection{Vorlagen für Veranstaltungstypen}
Das System kann einen zusätzlichen Mehrwert bieten, indem für ausgewählte Veranstaltungstypen schon Serviceslots vorangelegt werden. Diese sollen auch schon mit den bewährten Kategorien für den Typ der Veranstaltung vorbelegt sein.

Mit diesen Vorlagen werden die Hilfestellungen des Systems weiter ausgebaut, an denen sich ein Veranstalter orientieren kann. Er muss nur noch die Preisvorstellungen, den Zeitplan und die Dienstleister bestimmen.
